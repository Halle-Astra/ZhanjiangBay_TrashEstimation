% 中文CCT 论文模板
%%   %后面的内容表示注释
% 默认字体10pt,可选择11pt或12pt字体.
% 论文、书籍、报告、信件格式分别为:article, cbook, creport, cletter.
% 后面的内容表示注释

%%%%%%%%%%%%%%%%
%%%%%%%%%%%%%%%% 以下文章环境、宏包等定义

%\documentstyle[11pt]{carticle}
\documentclass[12pt]{ctexart}  %%   格式选用
\topmargin=-5mm \textwidth=148mm \textheight=235mm  %%   定义页边距s_n等
\usepackage[centertags]{amsmath}   %%  命令\usepackage{**}表示导入需要使用的宏包**,可通过百度,得到实现什么功能需要导入的各类宏包,在此不一一介绍
\usepackage{amsfonts}
\usepackage{amssymb}
\usepackage{amsthm}
\usepackage{booktabs}
\usepackage{enumerate}
\usepackage[mathscr]{eucal}
\usepackage{cite}
\usepackage{color}
\usepackage{caption}
\usepackage{graphicx}
\usepackage{float}
\usepackage{extarrows}
%\newenvironment{sequation}{\tiny\begin{table}}{\end{table}}
\setcounter{section}{3}
\renewcommand{\theequation}{\arabic{section}.\arabic{equation}}  %% 利用\renewcommand{}可以定义章节、定理、命题等的编号方式
%\renewcommand{\theequation}{\mbox{\arabic{section}.\arabic{equation}}}
\newtheorem{Theorem}{\qquad  定理}[section] %% 利用\newtheorem{}{}[]可以定义定理、定义、命题等的格式
\newtheorem{Coro}{\qquad 推论}[section]
\newtheorem{Definition}{\qquad 定义}[section]
%\newtheorem{Pro}[theorem]{\qquad 命题}[section]
\newtheorem{Lemma}{\qquad 引理}[section]
\renewcommand{\thefootnote}{\fnsymbol{footnote}}
%\renewcommand\sectionformat{\flushleft}
%\renewcommand\sectionname{}
\renewcommand\section{ 第\arabic{section}节}
\renewcommand\subsection{\arabic{section}.\arabic{subsection}}
\newcommand{\chuhao}{\fontsize{42pt}{\baselineskip}\selectfont}
\newcommand{\xiaochuhao}{\fontsize{36pt}{\baselineskip}\selectfont}
\newcommand{\yihao}{\fontsize{28pt}{\baselineskip}\selectfont}
\newcommand{\erhao}{\fontsize{21pt}{\baselineskip}\selectfont}
\newcommand{\xiaoerhao}{\fontsize{18pt}{\baselineskip}\selectfont}
\newcommand{\sanhao}{\fontsize{15.75pt}{\baselineskip}\selectfont}
\newcommand{\sihao}{\fontsize{14pt}{\baselineskip}\selectfont}
\newcommand{\xiaosi}{\fontsize{12pt}{\baselineskip}\selectfont}
\newcommand{\wuhao}{\fontsize{10.5pt}{\baselineskip}\selectfont}
\newcommand{\xiaowuhao}{\fontsize{9pt}{\baselineskip}\selectfont}
\newcommand{\liuhao}{\fontsize{7.875pt}{\baselineskip}\selectfont}
\newcommand{\qihao}{\fontsize{5.25pt}{\baselineskip}\selectfont}
\renewcommand{\tablename}{\wuhao\bf 表} %% 重新定义表头的大小
\renewcommand{\figurename}{\wuhao 图} %% 重新定义图尾的大小
\renewcommand{\refname}{\xiaosi\bf 参考文献} %% 重新定义参考文献标题的大小

%%%%%%%%%%%%%%%%%%%%%%%%%%%%%%%%%%%%%
%%%%%%%%%%%%%%%%%%%%%%%%%%%%%%%%%%%%%  以下开始文章撰写

\begin{document}
\thispagestyle{empty}  %% 表示本页无页码

%%以下输入题目,\sanhao表示三号,\bf表示黑体  \begin{center}  **  \end{center} 表示** 居中


\begin{center}
{\sanhao\bf 习题5.5}
\end{center}

利用对于沿岸地区的分段处理,根据沿岸地区每一点的水流进行对垃圾的运输,以及水流速度进行多点垃圾相交从而得出垃圾分布点,再通过聚类分析得到垃圾预估聚集,假设有两条汇聚的水流,设水流起源点A为$(x_1,y_1)$,水流起源点B为$(x_2,y_2)$,水流的交点C为$(x_3,y_3)$。

如图,由几何关系得:$\frac{CE}{BE}=\tan {\theta_2}=\frac {y_1-y_2}{x_3-x_2}$

~~~~~~~~~~~~~~~~~~~~~~~~~~~~~~~$\frac{CF}{AF}=\tan {\theta_1}=\frac {y_3-y_1}{x_3-x_1}$

~~~~~~~~~~~~~~~~~~~~~~~~~~~~~~~$\frac{CE}{DE}=\tan {\theta_1}=\frac {y_3-y_2}{x_3-x_D}$

又$\therefore y_2=\tan {\theta_1 x_2}+b$

~~~~~~~$y_3-y_2=\tan {\theta_2}(x_3-x_2)$

~~~~~~~~~~~$\Rightarrow y_3=\tan {\theta_2}(x_3-x_2)+y_2$时

由$\frac{CF}{AF}=\tan {\theta_1}$,有

~~~~~~~$\frac {y_3-y_1}{x_3-x_1}=\frac{\tan {\theta_2}(x_3-x_2)+y_2-y_1}{x_3-x_1}=\tan {\theta_1}$

~~~~~~~$\tan {\theta_2}(x_3-x_2)+y_2-y_1=\tan {\theta_1}(x_3-x_1)$

~~~~~~~~~~~~~~~~~~~~~~~$x_3(\tan {\theta_2}-\tan {\theta_1})-\tan {\theta_2}x_2+\tan {\theta_1}x_1+y_2y_1=0$

~~~~~~~~~~~~~~~~~~~~~~~$x_3=\frac{y_1-y_2-\tan {\theta_1}x_1+\tan {\theta_2}x_2}{\tan {\theta_2}-\tan {\theta_1}}$

$y_3=\tan {\theta_2}(x_3-x_2)+y_2$直接带入$X_3$,

最后得出水流聚集点$C $~$(x_3,y_3)$代值数学建模。








\end{document}
